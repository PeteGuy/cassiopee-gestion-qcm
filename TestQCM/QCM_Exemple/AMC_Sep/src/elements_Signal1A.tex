%%%%%%%%%%%%%%%%%%%%%%%%%%%%%%%%%%%%%%%%%%%%%%%%%%%%%%%%%%%%%%%%%%%%%%%%%%%%
\element{quNonClassee}{
  \begin{question}{} La fonction d'autocorrélation
    $\gamma_x(\tau),\tau\in\RR$ (en énergie ou puissance) d'un signal
    à temps continu $x(t),t\in\RR$:
   \begin{reponses}
   \mauvaise{ est toujours une fonction périodique.}
   \bonne{ peut {\^e}tre une fonction périodique selon le signal $x(t)$.}
   \mauvaise{ vérifie pour tout $\tau$: $\gamma_x(\tau)\geq0$.}
   \mauvaise{ est définie comme le module au carré de la transformée de
     Fourier de $x(t)$.}
   \end{reponses}
\end{question}
}
%%%%%%%%%%%%%%%%%%%%%%%%%%%%%%%%%%%%%%%%%%%%%%%%%%%%%%%%%%%%%%%%%%%%%%%%%%%%
\element{quNonClassee}{
  \begin{question}{} On peut observer un phénomène d'élargissement des
    raies spectrales:
  \begin{reponses}
    \mauvaise{ uniquement lorsque le principe d'incertitude de
      Heisenberg n'est pas contredit par le signal.}

    \mauvaise{ lorsque la fréquence d'échantillonnage est mal
      choisie.}

    \bonne{ de fa{\c c}on générale lors de la troncature temporelle
      d'un signal comportant des raies dans son spectre.}

    \mauvaise{ uniquement lorsque le signal étudié est une sinusoïde
      convoluée avec une porte.}
  \end{reponses}
\end{question}
}
%%%%%%%%%%%%%%%%%%%%%%%%%%%%%%%%%%%%%%%%%%%%%%%%%%%%%%%%%%%%%%%%%%%%%%%%%%%%
\element{quNonClassee}{
  \begin{questionmult}{} Soit $x(t)$ un signal et $X(f)$ sa
    transformée de Fourier (ce que l'on note par: $x(t) \TF
    X(f)$). Alors:
  \begin{reponses}
    \bonne{ $x(t)e^{\ic2\pi f_0t} \TF X(f-f_0)$}

    \mauvaise{ $x(t+t_0) \TF X(ft_0)$}

    \mauvaise{ $x(at) \TF aX(af)$}

    \bonne{ si $x(t)$ est réel, alors $X(f) = X(-f)^*$.}
  \end{reponses}
\end{questionmult}
}
%%%%%%%%%%%%%%%%%%%%%%%%%%%%%%%%%%%%%%%%%%%%%%%%%%%%%%%%%%%%%%%%%%%%%%%%%%%%
\element{quNonClassee}{
\begin{question}{} Un signal sinusoïdal pur de fréquence 418Hz est
  échantillonné. La durée entre deux échantillons est de 20ms.
  \begin{reponses}
  \mauvaise{ La condition d'échantillonnage de Shannon est respectée.}
  \mauvaise{ La condition d'échantillonnage de Shannon n'est pas respectée.
    Il y a repliement de spectre et une raie est repliée à 20Hz.}
  \bonne{ La condition d'échantillonnage de Shannon n'est pas respectée.
    Il y a repliement de spectre et une raie est repliée à 18Hz.}
  \mauvaise{ La transformée de Fourier à temps discret du signal
    échantillonné n'est pas définie car la condition d'échantillonnage
    n'est pas respectée.}
  \end{reponses}
\end{question}
}
%%%%%%%%%%%%%%%%%%%%%%%%%%%%%%%%%%%%%%%%%%%%%%%%%%%%%%%%%%%%%%%%%%%%%%%%%%%%
\element{quNonClassee}{
\begin{question}{} Soit $x(t)$ un signal réel et $X(f)$ sa transformée de
  Fourier. Rappelons que le signal analytique $x_a(t)$ associé au
  signal réel $x(t)$ peut {\^e}tre défini par sa transformée de Fourier:
   \[
   X_a(f) =
   \begin{cases}
     2X(f) & \text{ si } f\geq0, \\
     0    & \text{ si } f<0.
   \end{cases}
   \]
   \begin{reponses}
   \mauvaise{ La transformation de $x(t)$ en signal analytique $x_a(t)$ est
     indispensable avant toute analyse à l'analyseur de spectre car
     seules les fréquences positives existent.}
     \mauvaise{ Le signal analytique $x_a(t)$ est un signal réel.}
     \bonne{ Le signal analytique $x_a(t)$ est un signal complexe.}
     \mauvaise{ Le signal analytique $x_a(t)$ peut {\^e}tre réel ou complexe,
       celà dépend du signal $x(t)$.}
   \end{reponses}
\end{question}
}
%%%%%%%%%%%%%%%%%%%%%%%%%%%%%%%%%%%%%%%%%%%%%%%%%%%%%%%%%%%%%%%%%%%%%%%%%%%%
\element{quNonClassee}{
\begin{questionmult}{} L'algorithme de transformée de Fourier rapide (FFT):
  \begin{reponses}
  \mauvaise{est un algorithme rapide qui permet de calculer la transformée
    de Fourier discrète; il s'applique dès que le nombre
    d'échantillons est pair.}
  \bonne{est un algorithme rapide qui permet de calculer la transformée
    de Fourier discrète; il s'applique lorsque le nombre
    d'échantillons est une puissance de deux.}
  \mauvaise{ est un algorithme rapide pour le calcul du produit matriciel
    de l'équation (\ref{eq:prodFFT_1}) ci-dessous o{\`u} $x_1,\ldots,x_N$ sont
    $N$ échantillons; l'algorithme s'applique dès que $N$ est pair.}
  \bonne{est un algorithme rapide pour le calcul du produit matriciel
    de l'équation (\ref{eq:prodFFT_1}) ci-dessous o{\`u} $x_1,\ldots,x_N$ sont
    $N$ échantillons; l'algorithme s'applique lorsque $N$ est une
    puissance de deux.
    \begin{equation*}
      \label{eq:prodFFT_1}
      \begin{pmatrix}
        1      & 1        & 1         & 1      & 1    \\
        1      & w        & w^2       & \cdots & w^{N-1}    \\
        1      & w^2      & w^4       & \cdots & w^{2(N-1)} \\
        \vdots & \vdots   & \vdots    & \ddots &\vdots      \\
        1      &  w^{N-1} &w^{2(N-1)} & \cdots & w^{(N-1)^2}
      \end{pmatrix}
      \begin{pmatrix}
        x_1 \\ x_2 \\ x_3 \\ \vdots \\ x_N
      \end{pmatrix} \qquad \text{avec: } w=e^{\ic2\pi/N} \,.
    \end{equation*}}
  \end{reponses}
\end{questionmult}
}
%%%%%%%%%%%%%%%%%%%%%%%%%%%%%%%%%%%%%%%%%%%%%%%%%%%%%%%%%%%%%%%%%%%%%%%%%%%%
\element{quNonClassee}{
\begin{question}{} Un filtre à temps discret défini par sa réponse
  impulsionnelle $(h_n)_{n\in\ZZ}$ ou par sa transformée en $z$ $H[z]$
  est stable si et seulement si:
  \begin{reponses}
  \mauvaise{ le domaine de convergence de $H[z]$ est un disque de rayon $R\in\RR_+^*$.}
  \mauvaise{ le domaine de convergence de $H[z]$ est du type $\{z\in\CC \;|\;
    |z|>R\}$, o{\`u} $R\in\RR_+^*$.}
  \bonne{ l'ensemble $\{z\in\CC \;|\;|z|=1\}$ est inclus dans le domaine de 
    convergence de $H[z]$.}
  \mauvaise{ $\lim_{n\to+\infty} h_n = 0$.}
  \end{reponses}
\end{question}
}
%%%%%%%%%%%%%%%%%%%%%%%%%%%%%%%%%%%%%%%%%%%%%%%%%%%%%%%%%%%%%%%%%%%%%%%%%%%%
\element{quNonClassee}{
\begin{question}{} Si $x = \begin{bmatrix} 1& -2& 3& -4& 5& -6& 5&
    -2\end{bmatrix}$ et si on note $X = \begin{bmatrix} X_0& X_1& \ldots &
    X_7 \end{bmatrix}$ la transformée de Fourier discrète de la suite
  d'échantillons contenus dans le vecteur $x$, que vaut $X_0$ ?
  \begin{reponses}
  \mauvaise{ $+\sqrt{2} - i2\pi$}
  \mauvaise{ $-\sqrt{2} + i2\pi$}
  \bonne{ $0$}
  \mauvaise{ $8$}
  \end{reponses}
  \end{question}
}
%%%%%%%%%%%%%%%%%%%%%%%%%%%%%%%%%%%%%%%%%%%%%%%%%%%%%%%%%%%%%%%%%%%%%%%%%%%%
%%%%%%%%%%%%%%%%%%%%%%%%%%%%%%%%%%%%%%%%%%%%%%%%%%%%%%%%%%%%%%%%%%%%%%%%%%%%
\element{quNonClassee}{
\begin{questionmult}{} Soit $x(t)$ un signal déterministe d'énergie finie,
    $X(f)$ sa transformée de Fourier et $\Gamma_x(f)$ sa densité spectrale
    d'énergie.
    \begin{reponses}
    \bonne{On a: $\forall f\in\RR \quad \Gamma_x(f) = |X(f)|^2$.}
    \mauvaise{On a: $\forall f\in\RR \quad \Gamma_x(f) = |X(f)|$.}
    \bonne{On a l'égalité: $\ds \int_{-\infty}^\infty|x(t)|^2\,dt=\int_{-\infty}^\infty \Gamma_x(f) \,df$.}
    \mauvaise{$\Gamma_x(f)$ est toujours maximal en $0$.}
    \end{reponses}
\end{questionmult}
}
%%%%%%%%%%%%%%%%%%%%%%%%%%%%%%%%%%%%%%%%%%%%%%%%%%%%%%%%%%%%%%%%%%%%%%%%%%%%
\element{quNonClassee}{
\begin{question}{} La formule d'interpolation d'un signal à bande limitée:
  \begin{reponses}
  \mauvaise{ est une approximation qui permet d'approcher les valeurs du
    signal entre les échantillons.}
  \bonne{ est une égalité; la reconstruction exacte du signal entre deux
    échantillons est possible en théorie.}
  \mauvaise{ ne fait intervenir que les échantillons passés du signal car
    un filtre de restitution doit {\^e}tre causal.}
  \mauvaise{ n'est valable que pour des signaux périodiques.}
  \end{reponses}
\end{question}
}
%%%%%%%%%%%%%%%%%%%%%%%%%%%%%%%%%%%%%%%%%%%%%%%%%%%%%%%%%%%%%%%%%%%%%%%%%%%%
\element{quNonClassee}{
\begin{question}{} Un filtre à réponse impulsionnelle finie est aussi appelé:
  \begin{reponses}
  \bonne{ filtre transverse.}
  \mauvaise{ filtre récursif.}
  \mauvaise{ filtre à causalité finie.}
  \mauvaise{ filtre à p{\^o}les positifs.}
  \end{reponses}
\end{question}
}
%%%%%%%%%%%%%%%%%%%%%%%%%%%%%%%%%%%%%%%%%%%%%%%%%%%%%%%%%%%%%%%%%%%%%%%%%%%%
\element{quNonClassee}{
\begin{question}{} On considère l'opération qui à un signal à temps continu
   $x(t)$ associe le signal $y(t)$ défini par:
   \[
   y(t)=\int_{t}^{t+\alpha} x(\theta) \;d\theta \qquad \text{avec $\alpha>0$.}
   \]
   \begin{reponses}
   \mauvaise{ C'est une opération de filtrage par un filtre non causal dont
     la réponse impulsionnelle est donnée par:
     \[
     h(\theta)=
     \begin{cases}
       x(\theta) & \text{si } \theta\in[t,t+\alpha] \\
       0 & \text{sinon.}
     \end{cases}
     \]}
   \mauvaise{ C'est une opération de filtrage par un filtre causal dont la
     réponse impulsionnelle est donnée par:
     \[
     h(\theta)= 
     \begin{cases}
       x(\theta) & \text{si } \theta\in[t,t+\alpha] \\
       0 & \text{sinon.}
     \end{cases}
     \]}
   \bonne{ C'est une opération de filtrage par un filtre non causal dont
     la réponse impulsionnelle est donnée par:
     \[
     h(\theta)=
     \begin{cases}
       1 & \text{si } \theta\in[-\alpha,0] \\
       0 & \text{sinon.}
     \end{cases}
     \]}
   \mauvaise{ C'est une opération de filtrage par un filtre causal dont la
     réponse impulsionnelle est donnée par:
     \[
     h(\theta)=
     \begin{cases}
       1 & \text{si } \theta\in[-\alpha,0] \\
       0 & \text{sinon.}
     \end{cases}
     \]}
   \end{reponses}
\end{question}
}
%%%%%%%%%%%%%%%%%%%%%%%%%%%%%%%%%%%%%%%%%%%%%%%%%%%%%%%%%%%%%%%%%%%%%%%%%%%%
\element{quNonClassee}{
\begin{question}{} Soit $T\in\RR_+$ et $p_T(t)$ le signal porte défini par
   $p_T(t) = 1$ si $t\in[-T/2,T/2]$ et $p_T(t)=0$ si $t\notin[-T/2,T/2]$.  La
   transformée de Fourier $P_T(f)$ de $p_T(t)$:
   \begin{reponses}
   \mauvaise{n'est pas dérivable car le signal $p_T(t)$ n'est pas continu.}
   \bonne{vérifie l'égalité: $\displaystyle \int_{-\infty}^{+\infty} |P_T(f)|^2 \,df
     = T $.}
   \mauvaise{est à valeurs complexes (et non pas réelles), comme c'est le
     cas pour toutes les transformées de Fourier.}
   \mauvaise{vaut: $\displaystyle P_T(f) = T\left(\frac{\sin(\pi fT)}{\pi
         fT}\right)^2$ si $f\neq0$ et $P_T(0)=1$.}
   \end{reponses}
\end{question}
}
%%%%%%%%%%%%%%%%%%%%%%%%%%%%%%%%%%%%%%%%%%%%%%%%%%%%%%%%%%%%%%%%%%%%%%%%%%%%
\element{quNonClassee}{
\begin{question}{} Soit $x(t)$ un signal à bande limitée, dont le support de
   la transformée de Fourier est inclus dans $[-B,B]$; soit $h(t)$ la
   réponse impulsionnelle d'un filtre quelconque. Soit $y(t)$ la
   sortie de ce filtre excité par $x(t)$.
   \begin{reponses}
   \bonne{ $y(t)$ est un signal à bande limitée.}
   \mauvaise{ si $T$ est une période d'échantillonnage telle que $1/T>2B$,
     alors le signal à temps discret $(y(nT))$ co{\"\i}ncide avec la
     version filtrée de $(x(nT))$, la réponse impulsionnelle du filtre
     numérique dont il est question étant $(h(nT))$.}
   \mauvaise{ puisque $\displaystyle y(t) = \int_{\mathbb{R}} h(\tau)x(t-\tau)d\tau$,
     on peut toujours écrire, quel que soit $T>0$:
     \[
     y(nT) = \sum_{k\in\ZZ} h(kT)x(nT-kT)
     \]}
   \mauvaise{ $y(t)$ est un signal causal car obtenu par une opération de
     filtrage.}
   \end{reponses}
\end{question}
}
%%%%%%%%%%%%%%%%%%%%%%%%%%%%%%%%%%%%%%%%%%%%%%%%%%%%%%%%%%%%%%%%%%%%%%%%%%%%
\element{quNonClassee}{
\begin{question}{} Un filtre numérique rationnel défini par sa fonction de
   transfert en $z$ $H[z]$ ou par sa réponse impulsionnelle
   $(h_n)_{n\in\ZZ}$ est stable si et seulement si:
   \begin{reponses}
   \mauvaise{ le domaine de convergence de $H[z]$ est du type
     $\{z\in\CC \,|\, |z|>R\}$, o{\`u} $R\in\RR_+^*$.}
   \mauvaise{ les p{\^o}les de $H[z]$ sont à partie réelle négative.}
   \mauvaise{ la réponse impulsionnelle est bornée (càd il existe
     $M\in\RR^*_+$ tel que pour tout $n$, $|h_n|<M$).}
   \bonne{ l'ensemble $\{z\in\CC/|z|=1\}$ est inclus dans le domaine de 
       convergence de $H[z]$.}
   \end{reponses}
\end{question}
}
%%%%%%%%%%%%%%%%%%%%%%%%%%%%%%%%%%%%%%%%%%%%%%%%%%%%%%%%%%%%%%%%%%%%%%%%%%%%
\element{quNonClassee}{
\begin{question}{} Soit $x(t)$ un signal déterministe d'énergie finie, $X(f)$
  sa transformée de Fourier et $\Gamma_x(f)$ sa densité spectrale
  d'énergie.
  \begin{reponses}
  \bonne{ L'énergie de $x(t)$ vaut $\ds \int_{-\infty}^\infty |X(f)|^2 \,df$.}
  \mauvaise{ On a: $\forall f\in\RR \quad \Gamma_x(f) = |X(f)|$.}
  \mauvaise{ $\Gamma_x(f)$ admet toujours une symétrie hermitienne, càd: $\forall f\in\RR \quad \Gamma(f)=\Gamma(-f)^*$.}
  \mauvaise{ $\Gamma_x(f)$ est toujours maximale en $0$.}
  \end{reponses}
\end{question}
}
%%%%%%%%%%%%%%%%%%%%%%%%%%%%%%%%%%%%%%%%%%%%%%%%%%%%%%%%%%%%%%%%%%%%%%%%%%%%
\element{quNonClassee}{
\begin{question}{} On souhaite tracer sous Matlab la courbe de la fonction
  $t\mapsto e^{-t}\sin(2\pi t)$ pour $t\in[-2,2]$. Laquelle de ces suites
  d'instructions Matlab permet-elle d'obtenir un tracé d'allure
  correcte?
  \begin{reponses}
  \mauvaise{ \texttt{plot(exp(-t)*sin(2*pi*t),t=[-2..2]);}}
  \mauvaise{ \texttt{t = [-2:0.05:2]; x = (exp(-t)*sin(2*pi*t)); plot(t,x);}}
  \bonne{ \texttt{t = [-2:0.05:2]; x = (exp(-t).*sin(2*pi*t)); plot(t,x);}}
  \mauvaise{ \texttt{for t=-2:0.01:2, \\ plot(exp(-t).*sin(2*pi*t)); \\ end;}}
  \end{reponses}
\end{question}
}
%%%%%%%%%%%%%%%%%%%%%%%%%%%%%%%%%%%%%%%%%%%%%%%%%%%%%%%%%%%%%%%%%%%%%%%%%%%%
\element{quNonClassee}{
\begin{question}{} Sous le logiciel \textsc{Matlab}, on entre les
  commandes suivantes:
  \texttt{t = (1:512); x = cos(2*pi*0.25*t); X = fft(x);}
  \begin{reponses}
   \bonne{ La variable \texttt{x} est un vecteur de taille 512 qui
     contient des échantillons d'une sinuso\"ide de fréquence 0.25
     échantillonnée à la période 1.}
   \mauvaise{ La commande \texttt{plot(t,abs(X))} permet de tracer le module
     de la transformée de Fourier discrète de \texttt{x} en fonction de
     la fréquence normalisée.}
   \mauvaise{ La commande \texttt{plot(t/512,abs(X))} affiche le
     module de la transformée de Fourier discrète de \texttt{x} en
     fonction de la fréquence normalisée.}
   \mauvaise{ La commande \texttt{plot(fft(X))} affiche le module de la
     transformée de Fourier discrète de \texttt{x} en fonction de la
     fréquence normalisée.}
  \end{reponses}
\end{question}
}
%%%%%%%%%%%%%%%%%%%%%%%%%%%%%%%%%%%%%%%%%%%%%%%%%%%%%%%%%%%%%%%%%%%%%%%%%%%%
\element{quNonClassee}{
\begin{questionmult}{} Sous le logiciel \textsc{Matlab}, on suppose avoir chargé
  dans la variable \texttt{x} des échantillons d'un signal prélevés à
  une fréquence d'échantillonnage de $1\text{kHz}$.
  \begin{reponses}
  \mauvaise{ Les commandes \texttt{N=length(x); plot((1:N),abs(fft(x)));}
    tracent le module de la transformée de Fourier à temps discret du
    vecteur \texttt{x}. La fréquence réduite (ou normalisée) est
    indiquée est abscisse.}
  \bonne{ Les commandes \texttt{N=length(x);
      plot((0:N-1)/N,abs(fft(x)));} tracent le module de la
    transformée de Fourier à temps discret du vecteur \texttt{x}.  La
    fréquence réduite (ou normalisée) est indiquée est abscisse.}
  \mauvaise{ Les commandes \texttt{N=length(x);
      plot((0:N-1)/N*1000,abs(fft(x)));} tracent le module de la
    transformée de Fourier à temps discret du vecteur \texttt{x}.  La
    fréquence réduite (ou normalisée) est indiquée est abscisse.}
  \bonne{ Les commandes \texttt{N=length(x);
      plot((0:N-1)/N*1000,abs(fft(x)));} tracent le module de la
    transformée de Fourier à temps discret du vecteur \texttt{x}. La
    fréquence réelle est indiquée est abscisse.}
  \end{reponses}
\end{questionmult}
}
%%%%%%%%%%%%%%%%%%%%%%%%%%%%%%%%%%%%%%%%%%%%%%%%%%%%%%%%%%%%%%%%%%%%%%%%%%%%
\element{quNonClassee}{
\begin{questionmult}{} Un filtre numérique défini par sa fonction de transfert
    en $z$ $H[z]$ ou par sa réponse impulsionnelle $(h_n)_{n\in\ZZ}$ est
    causal si et seulement si:
    \begin{reponses}
    \mauvaise{ $h_n>0$ pout tout $n>0$.}
    \bonne{ $h_n=0$ pour tout $n<0$.}
    \bonne{ le domaine de convergence de $H[z]$ est du type $\{z\in\CC \;|\;
      |z|>R\}\cup\{\infty\}$ (càd le complémentaire d'un disque centré en $0$, point à l'infini compris).}
    \mauvaise{ le domaine de convergence de $H[z]$ est du type $\{z\in\CC \;|\;
      R_1<|z|<R_2\}$ o{\`u} $R_2$ est un réel positif (càd un anneau compris entre les cercles centrés en $0$ et de rayon $R_1$ et $R_2$).}
  \end{reponses}
\end{questionmult}
}
%%%%%%%%%%%%%%%%%%%%%%%%%%%%%%%%%%%%%%%%%%%%%%%%%%%%%%%%%%%%%%%%%%%%%%%%%%%%
\element{quNonClassee}{
\begin{question}{} Soit $\Gamma_x(f)$ la densité spectrale d'énergie
    d'un signal $x(t)$.
    \begin{reponses}
    \mauvaise{ L'énergie du signal vaut $\Gamma_x(0)$.}
    \mauvaise{ L'énergie du signal vaut $|\Gamma_x(0)|^2$.}
    \bonne{ L'énergie du signal vaut $\int_\RR \Gamma_x(f) \; df$.}
    \mauvaise{ L'énergie du signal vaut $\int_\RR |\Gamma_x(f)|^2 \; df$.}
    \end{reponses}
\end{question}
}
%%%%%%%%%%%%%%%%%%%%%%%%%%%%%%%%%%%%%%%%%%%%%%%%%%%%%%%%%%%%%%%%%%%%%%%%%%%%
\element{quNonClassee}{
\begin{question}{} Le signal $x(t)=\cos(2\pi f_0t)$ avec $f_0=93\mathrm{Hz}$ est
    échantillonné à la fréquence d'échantillonnage $F_e=100\mathrm{Hz}$ pour
    former le signal $x_n=x(\frac{n}{F_e}),n\in\ZZ$.
    \begin{reponses}
    \mauvaise{ Il n'est pas possible de procéder ainsi car $2f_0>F_e$ et la
      condition de Shannon-Nyquist du théorème d'échantillonnage n'est
      pas vérifiée.}
    \mauvaise{ Il est possible de procéder ainsi car $f_0\leq F_e$ et la
      condition de Shannon-Nyquist du théorème d'échantillonnage est
      vérifiée.}
    \mauvaise{ Indépendamment de $f_0$ et $F_e$, il est toujours possible de
      procéder ainsi. Ici, $f_0\leq F_e$ et la condition de Shannon-Nyquist
      du théorème d'échantillonnage est donc vérifiée.}
    \bonne{ Indépendamment de $f_0$ et $F_e$, il est toujours possible de
      procéder ainsi. Ici, $2f_0>F_e$ et la condition de Shannon-Nyquist
      du théorème d'échantillonnage n'est donc pas vérifiée.}
    \end{reponses}
\end{question}
}
%%%%%%%%%%%%%%%%%%%%%%%%%%%%%%%%%%%%%%%%%%%%%%%%%%%%%%%%%%%%%%%%%%%%%%%%%%%%
\element{quNonClassee}{
\begin{question}{} L'algorithme de transformée de Fourier rapide (FFT):
   \begin{reponses}
   \mauvaise{ est un algorithme rapide basé sur les propriétés fondamentales
     de la transformée de Fourier (linéarité, changement de variables,
     Parseval,\ldots). Il permet le calcul de la transformée de Fourier des
     signaux à temps continu.}
   \mauvaise{ est un algorithme rapide basé sur le théorème des résidus et
     qui permet le calcul de la transformée de Fourier des signaux à
     temps continu.}
   \bonne{ est un algorithme rapide pour le calcul du produit matriciel
     de l'équation (\ref{eq:prodFFT}) ci-dessous lorsque $N$ est une
     puissance de deux.
     \begin{equation}
       \label{eq:prodFFT}
       \begin{pmatrix}
         1      & 1        & 1         & 1      & 1    \\
         1      & w        & w^2       & \cdots & w^{N-1}    \\
         1      & w^2      & w^4       & \cdots & w^{2(N-1)} \\
         \vdots & \vdots   & \vdots    & \ddots &\vdots      \\
         1      &  w^{N-1} &w^{2(N-1)} & \cdots & w^{(N-1)^2}
       \end{pmatrix}
       \begin{pmatrix}
         x_1 \\ x_2 \\ x_3 \\ \vdots \\ x_N
       \end{pmatrix} \qquad \text{avec: } w=e^{\ic2\pi/N} 
     \end{equation}}
   \mauvaise{ est un algorithme rapide basé sur les propriétés du filtrage à
     temps continu et qui est utilisé dans les analyseurs de spectre
     analogiques.}
   \end{reponses}
\end{question}
}
%%%%%%%%%%%%%%%%%%%%%%%%%%%%%%%%%%%%%%%%%%%%%%%%%%%%%%%%%%%%%%%%%%%%%%%%%%%%
\element{quNonClassee}{
\begin{question}{} Sur un analyseur de spectre analogique (tel que celui que
 vous avez manipulé en TP):
 \begin{reponses}
   \mauvaise{ une durée de balayage plus faible améliore toujours la
     résolution.}
   \bonne{ une durée de balayage plus grande permet de choisir une bande
     passante plus faible du filtre d'analyse et d'améliorer la
     résolution.}
   \mauvaise{ une bonne résolution est obtenue avec une bande passante
     large du filtre d'analyse.}
   \mauvaise{ la résolution en fréquence dépend de la rapidité à laquelle
     varie le signal et pas de l'appareil.}
 \end{reponses}
\end{question}
}
%%%%%%%%%%%%%%%%%%%%%%%%%%%%%%%%%%%%%%%%%%%%%%%%%%%%%%%%%%%%%%%%%%%%%%%%%%%%
\element{quNonClassee}{
\begin{question}{} Que renvoie la commande \texttt{fftshift([1 1 1 1])} sous
  le logiciel \textsc{Matlab}?
  \begin{reponses}
    \mauvaise{ Le vecteur \texttt{[1 2 3 4]}.}
    \mauvaise{ Le vecteur \texttt{[4 0 0 0]}.}
    \mauvaise{ Le vecteur \texttt{[1+i 1-i 1+i 1-i]}.}
    \bonne{ Le vecteur \texttt{[1 1 1 1]}.}
  \end{reponses}
\end{question}
}
%%%%%%%%%%%%%%%%%%%%%%%%%%%%%%%%%%%%%%%%%%%%%%%%%%%%%%%%%%%%%%%%%%%%%%%%%%%%
\element{quNonClassee}{
\begin{question}{} Que renvoie la commande \texttt{fft([1 0 0 0])} sous le
  logiciel \textsc{Matlab}?
  \begin{reponses}
    \mauvaise{ Le vecteur \texttt{[1 2 3 4]}.}
    \mauvaise{ Le vecteur \texttt{[4 0 0 0]}.}
    \mauvaise{ Le vecteur \texttt{[1+i 1-i 1+i 1-i]}.}
    \bonne{ Le vecteur \texttt{[1 1 1 1]}.}
  \end{reponses}
\end{question}
}
%%%%%%%%%%%%%%%%%%%%%%%%%%%%%%%%%%%%%%%%%%%%%%%%%%%%%%%%%%%%%%%%%%%%%%%%%%%%
\element{quNonClassee}{
\begin{question}{} Que renvoie la commande \texttt{fft([1 1 1 1])} sous
  le logiciel \textsc{Matlab}?
  \begin{reponses}
    \mauvaise{ Le vecteur \texttt{[1-i 1+i 1-i 1+i]}.}
    \bonne{ Le vecteur \texttt{[4 0 0 0]}.}
    \mauvaise{ Le vecteur \texttt{[1+i 1-i 1+i 1-i]}.}
    \mauvaise{ Le vecteur \texttt{[1 1 1 1]}.}
  \end{reponses}
\end{question}
}
%%%%%%%%%%%%%%%%%%%%%%%%%%%%%%%%%%%%%%%%%%%%%%%%%%%%%%%%%%%%%%%%%%%%%%%%%%%%
\element{quNonClassee}{
\begin{question}{} Que renvoie la commande \texttt{fftshift([1 2 3 4])} sous
  le logiciel \textsc{Matlab}?
  \begin{reponses}
    \bonne{ Le vecteur \texttt{[3 4 1 2]}.}
    \mauvaise{ Le vecteur \texttt{[1 3 2 4]}.}
    \mauvaise{ La transformée de Fourier discrète du vecteur \texttt{[1 2 3
      4]}, avec des fréquences croissantes de $-\frac{1}{2}$ à
    $\frac{1}{2}$.}
    \mauvaise{ La transformée de Fourier discrète du vecteur \texttt{[1 2 3
      4]}, avec des fréquences croissantes de $0$ à $1$.}
  \end{reponses}
\end{question}
}
%%%%%%%%%%%%%%%%%%%%%%%%%%%%%%%%%%%%%%%%%%%%%%%%%%%%%%%%%%%%%%%%%%%%%%%%%%%%
\element{quNonClassee}{
\begin{question}{} Que renvoie la commande \texttt{fft([0 0 0 1]))} sous le logiciel \textsc{Matlab}?
  \begin{reponses}
    \mauvaise{ Le vecteur \texttt{[1-i 1+i 1-i 1+i]}.}
    \mauvaise{ Le vecteur \texttt{[4 0 0 0]}.}
    \mauvaise{ Le vecteur \texttt{[0 1 0 0]}.}
    \mauvaise{ Le vecteur \texttt{[1 1 1 1]}. }   
    \mauvaise{ Le vecteur \texttt{[1+i 1-i 1+i 1-i]}.}
    \bonne{ Le vecteur \texttt{[1 i -1 -i]}.}
  \end{reponses}
\end{question}
}
%%%%%%%%%%%%%%%%%%%%%%%%%%%%%%%%%%%%%%%%%%%%%%%%%%%%%%%%%%%%%%%%%%%%%%%%%%%%
\element{quNonClassee}{
\begin{question}{} Que renvoie la commande \texttt{fftshift([0 0 0 1]))} sous le logiciel \textsc{Matlab}?
  \begin{reponses}
    \mauvaise{ Le vecteur \texttt{[1-i 1+i 1-i 1+i]}.}
    \mauvaise{ Le vecteur \texttt{[4 0 0 0]}.}
    \bonne{ Le vecteur \texttt{[0 1 0 0]}.}
    \mauvaise{ Le vecteur \texttt{[1 1 1 1]}.}
    \mauvaise{ Le vecteur \texttt{[1+i 1-i 1+i 1-i]}.}
    \mauvaise{ Le vecteur \texttt{[1 i -1 -i]}.}
  \end{reponses}
\end{question}
}
%%%%%%%%%%%%%%%%%%%%%%%%%%%%%%%%%%%%%%%%%%%%%%%%%%%%%%%%%%%%%%%%%%%%%%%%%%%%
\element{quNonClassee}{
\begin{question}{} Que renvoie la commande \texttt{fft(fftshift([1 1 1 1]))} sous le logiciel \textsc{Matlab}?
  \begin{reponses}
    \mauvaise{ Le vecteur \texttt{[1-i 1+i 1-i 1+i]}.}
    \bonne{ Le vecteur \texttt{[4 0 0 0]}.}
    \mauvaise{ Le vecteur \texttt{[1+i 1-i 1+i 1-i]}.}
    \mauvaise{ Le vecteur \texttt{[1 1 1 1]}.}
  \end{reponses}
\end{question}
}
%%%%%%%%%%%%%%%%%%%%%%%%%%%%%%%%%%%%%%%%%%%%%%%%%%%%%%%%%%%%%%%%%%%%%%%%%%%%
\element{quNonClassee}{
\begin{question}{} Que renvoie la commande \texttt{fftshift(fft([1 1 1 1]))}
  sous le logiciel \textsc{Matlab}?
  \begin{reponses}
    \bonne{ Le vecteur \texttt{[0 0 4 0]}.}
    \mauvaise{ Le vecteur \texttt{[4 0 0 0]}.}
    \mauvaise{ Le vecteur \texttt{[1+i 1-i 1+i 1-i]}.}
    \mauvaise{ Le vecteur \texttt{[1-i 1+i 1-i 1+i]}.}
  \end{reponses}
\end{question}
}
%%%%%%%%%%%%%%%%%%%%%%%%%%%%%%%%%%%%%%%%%%%%%%%%%%%%%%%%%%%%%%%%%%%%%%%%%%%%
\element{quNonClassee}{
\begin{questionmult}{} Soit $\gamma_x(t)$ la fonction d'autocorrélation en puissance
  d'un signal $x(t)$ de puissance finie.
  \begin{reponses}
  \mauvaise{ $\gamma_x(t)\geq 0$ pour tout $t$.}
  \mauvaise{ La puissance de $x(t)$ vaut $\int_{-\infty}^{+\infty} |\gamma_x(t)|^2 \,dt$.}
  \mauvaise{ L'énergie de $x(t)$ vaut $\int_{-\infty}^{+\infty} |\gamma_x(t)|^2 \,dt$.}
  \bonne{ $\gamma_x(t)$ peut {\^e}tre une fonction périodique.}
  \bonne{ $\gamma_x(0)\geq|\gamma_x(t)|$ pour tout $t$.}
  \end{reponses}
\end{questionmult}
}
%%%%%%%%%%%%%%%%%%%%%%%%%%%%%%%%%%%%%%%%%%%%%%%%%%%%%%%%%%%%%%%%%%%%%%%%%%%%
\element{quNonClassee}{
  \begin{questionmult}{} Soit $\Gamma_x(f)$ la densité spectrale de
    puissance d'un signal $x(t)$ de puissance finie.
  \begin{reponses}
    \mauvaise{$\Gamma_x(0)$ est égal à la puissance du signal.}
    \bonne{$\Gamma_x(f)$ est positif.}
    \mauvaise{$\forall f\in\RR \quad \Gamma_x(f) \leq \Gamma_x(0)$.}
    \mauvaise{$\Gamma_x(f)$ n'existe que pour un signal $x(t)$ réel.}
    \mauvaise{La puissance du signal vaut $\int_\RR |\Gamma_x(f)|^2 \; df$.}
    \bonne{L'énergie (ou respectivement la puissance) du signal vaut
     $\int_\RR \Gamma_x(f) \; df$.}
    \mauvaise{L'énergie (ou respectivement la puissance) du signal vaut
    $\Gamma_x(0)$.}
     \mauvaise{L'énergie (ou respectivement la puissance)du signal vaut
     $|\Gamma_x(0)|^2$.}
    \end{reponses}
\end{questionmult}
}
%%%%%%%%%%%%%%%%%%%%%%%%%%%%%%%%%%%%%%%%%%%%%%%%%%%%%%%%%%%%%%%%%%%%%%%%%%%%
\element{quNonClassee}{
\begin{questionmult}{} Pour définir une densité spectrale de puissance d'un
  signal aléatoire, il faut:
  \begin{reponses}
    \mauvaise{ que toutes ses trajectoires soient d'énergie finie.}    
    \bonne{ lorsque ce signal est stationnaire au sens strict.}
    \bonne{ qu'il soit stationnaire au sens large.}
    \mauvaise{ que le module de sa transformée de Fourier soit borné.}
    \mauvaise{ que sa transformée de Fourier soit ergodique.}
  \end{reponses}
\end{questionmult}
}
%%%%%%%%%%%%%%%%%%%%%%%%%%%%%%%%%%%%%%%%%%%%%%%%%%%%%%%%%%%%%%%%%%%%%%%%%%%%
\element{quNonClassee}{
\begin{question}{} $(y_n)_{n\in\ZZ}$ est la sortie d'un filtre stable de
  réponse en fréquence $H(f)$ excité en entrée par un signal
  $(x_n)_{n\in\ZZ}$ aléatoire stationnaire au sens large.
  \begin{reponses}
    \mauvaise{ $(y_n)_{n\in\ZZ}$ est un signal aléatoire stationnaire au sens
    large et sa transformée de Fourier à temps discret est $H(f)X(f)$
    (o{\`u} $X(f)$ est la transformée de Fourier du signal aléatoire
    $(x_n)_{n\in\ZZ}$).}
    \mauvaise{ $(y_n)_{n\in\ZZ}$ est un signal aléatoire déterministe car le filtre est stable et sa transformée de Fourier à temps discret est $H(f)X(f)$
    (o{\`u} $X(f)$ est la transformée de Fourier du signal aléatoire
    $(x_n)_{n\in\ZZ}$).}
    \bonne{ $(y_n)_{n\in\ZZ}$ est un signal aléatoire stationnaire au sens
    large et sa densité spectrale de puissance est $|H(f)|^2\Gamma_x(f)$
    (o{\`u} $\Gamma_x(f)$ est la densité spectrale de puissance de
    $(x_n)_{n\in\ZZ}$).}
    \mauvaise{ $(y_n)_{n\in\ZZ}$ est un signal aléatoire stationnaire au sens
    large. En tant que signal aléatoire, on ne peut pas définir sa
    densité spectrale de puissance.}
  \end{reponses}
\end{question}
}
%%%%%%%%%%%%%%%%%%%%%%%%%%%%%%%%%%%%%%%%%%%%%%%%%%%%%%%%%%%%%%%%%%%%%%%%%%%%
\element{quNonClassee}{
\begin{question}{} On effectue l'analyse d'un signal à l'analyseur de spectre
  analogique (tel que celui que vous avez manipulé en TP). On garde
  constante la bande de fréquences
  $[f_{\mathrm{min}},f_{\mathrm{max}}]$ étudiée et affichée à
  l'analyseur.
  \begin{reponses}
    \mauvaise{ Pour améliorer la résolution en fréquence, il faut choisir
    un balayage plus rapide.}
    \mauvaise{ Pour améliorer la résolution en fréquence, il faut que le
    filtre d'analyse de l'analyseur soit moins sélectif et ait une
    bande passante plus large.}
    \bonne{ Pour un affichage plus rapide, il faut un balayage plus
    rapide. La vitesse de balayage possible dépend de la résolution
    choisie.}
    \mauvaise{ Pour un affichage plus rapide, il faut un balayage plus
    rapide. La vitesse de balayage possible peut {\^e}tre choisie
    indépendamment de la résolution.}
  \end{reponses}
\end{question}
}
%%%%%%%%%%%%%%%%%%%%%%%%%%%%%%%%%%%%%%%%%%%%%%%%%%%%%%%%%%%%%%%%%%%%%%%%%%%%
\element{quNonClassee}{
\begin{question}{} $(x_n)_{n\in\ZZ}$ est un bruit blanc de puissance $\sigma^2$
  envoyé en entrée d'un filtre stable de réponse impulsionnelle
  $(h_n)_{n\in\ZZ}$ et de réponse en fréquence $H(f)$ .
  \begin{reponses}
    \mauvaise{ La densité spectrale de puissance en sortie est $|H(f)|\sigma$.}
    \mauvaise{ La densité spectrale de puissance en sortie est $H(f)X(f)$
    o{\`u} $X(f)$ est la transformée de Fourier à temps discret de
    $(x_n)_{n\in\ZZ}$.}
    \mauvaise{ La densité spectrale de puissance en sortie est $H(f)X(f)$
    o{\`u} $X(f)$ est la transformée de Fourier rapide de $(x_n)_{n\in\ZZ}$.}
    \bonne{ La densité spectrale de puissance en sortie est
    $|H(f)|^2\sigma^2$.}
    \mauvaise{ La densité spectrale de puissance en sortie est $H(f)\sigma$.}
  \end{reponses}
\end{question}
}
%%%%%%%%%%%%%%%%%%%%%%%%%%%%%%%%%%%%%%%%%%%%%%%%%%%%%%%%%%%%%%%%%%%%%%%%%%%%
\element{quNonClassee}{
\begin{question}{} Un bruit blanc numérique:
  \begin{reponses}
    \bonne{ a une densité spectrale de puissance constante.}
    \mauvaise{ a une densité spectrale de puissance égale à un Dirac.}
    \mauvaise{ a pour transformée de Fourier une constante.}
    \mauvaise{ a pour transformée de Fourier un Dirac.}
  \end{reponses}
\end{question}
}
%%%%%%%%%%%%%%%%%%%%%%%%%%%%%%%%%%%%%%%%%%%%%%%%%%%%%%%%%%%%%%%%%%%%%%%%%%%%
\element{quNonClassee}{
\begin{question}{} \label{qu:filtreNum020609bis} La figure
  \ref{fig:filtreNum020609bis} représente schématiquement le module de la
  réponse en fréquence d'un filtre numérique en fonction de la
  fréquence normalisée.
   % \begin{figure}[htbp]
   %   \centering
   %   \begin{pspicture}(14,1.7)(0,0)
   %     %\psgrid[subgriddiv=1,griddots=5,gridlabels=7pt,gridcolor=red](0,-1)(15,20)
   %      %\psaxes[Ox=0,Dx=1,Oy=0,Dy=1,labels=none,ticks=none]{->}(0,0)(0,-1.5)(5.5,2)
   %     \put(5.5,0){
   %       \psaxes[Dx=1,dx=4,ticks=x,labels=x]{->}(0,0)(0,0)(4.5,1.5)
   %       \psline[linewidth=0.01,fillstyle=none](1.2,0)(1.4,1.5)(2.6,0.8)(2.8,0)
   %       \uput[45](4.5,0){$f$ (fréq. normalisée)} \uput[0](0,1.5){$|H(f)|$} }
   %   \end{pspicture}
   %   \caption{Module de la réponse en fréquence du filtre question
   %     \ref{qu:filtreNum020609bis}}
   %   \label{fig:filtreNum020609bis}
   % \end{figure}
  \begin{reponses}
    \mauvaise{ il s'agit d'un filtre passe-bande et sa réponse
    impulsionnelle est à valeurs complexes.}
    \mauvaise{ il s'agit d'un filtre coupe-bande et sa réponse
    impulsionnelle est à valeurs réelles.}
    \bonne{ il s'agit d'un filtre passe-haut et sa réponse
    impulsionnelle est à valeurs complexes.}
    \mauvaise{ il s'agit d'un filtre passe-haut et sa réponse
    impulsionnelle est à valeurs réelles.}
  \end{reponses}
\end{question}
}
%%%%%%%%%%%%%%%%%%%%%%%%%%%%%%%%%%%%%%%%%%%%%%%%%%%%%%%%%%%%%%%%%%%%%%%%%%%%
\element{quNonClassee}{
\begin{questionmult}{} Soit $\Gamma_x(f)$ la densité spectrale d'énergie (ou
  respectivement de puissance) d'un signal $x(t)$.
  \begin{reponses}
   \mauvaise{ $\Gamma_x(0)$ est égal à l'énergie (ou respectivement la puissance) du signal.}
   \bonne{ $\Gamma_x(f)$ est toujours positif.}
   \mauvaise{ $\forall f\in\RR \quad \Gamma_x(f) \leq \Gamma_x(0)$.}
   \mauvaise{ $\Gamma_x(f)$ n'existe que pour un signal $x(t)$ réel.}
  \mauvaise{ L'énergie (ou respectivement la puissance) du signal vaut
    $\int_\RR |\Gamma_x(f)|^2 \; df$.}
  \bonne{ L'énergie (ou respectivement la puissance) du signal vaut
    $\int_\RR \Gamma_x(f) \; df$.}
  \mauvaise{ L'énergie (ou respectivement la puissance) du signal vaut
    $\Gamma_x(0)$.}
  \mauvaise{ L'énergie (ou respectivement la puissance)du signal vaut
    $|\Gamma_x(0)|^2$.}
  \end{reponses}
\end{questionmult}
}
%%%%%%%%%%%%%%%%%%%%%%%%%%%%%%%%%%%%%%%%%%%%%%%%%%%%%%%%%%%%%%%%%%%%%%%%%%%%
\element{quNonClassee}{
\begin{questionmult}{} Un filtre numérique rationnel défini par sa fonction de
   transfert en $z$ $H[z]$ ou par sa réponse impulsionnelle
   $(h_n)_{n\in\ZZ}$ est stable si et seulement si:
   \begin{reponses}
   \mauvaise{ le domaine de convergence de $H[z]$ est un disque de rayon
     $R\in\RR_+^*$.}
   \mauvaise{ le domaine de convergence de $H[z]$ est du type
     $\{z\in\CC \,|\, |z|>R\}$, o{\`u} $R\in\RR_+^*$.}
   \mauvaise{ les p{\^o}les de $H[z]$ sont à partie réelle négative.}
   \mauvaise{ la réponse impulsionnelle est bornée (càd il existe
     $M\in\RR^*_+$ tel que pour tout $n$, $|h_n|<M$).}
   \bonne{ $\sum_{k=-\infty}^{+\infty} |h_k|$ est fini.}
   \bonne{ l'ensemble $\{z\in\CC/|z|=1\}$ est inclus dans le domaine de 
       convergence de $H[z]$.}
  \mauvaise{ $\lim_{n\to+\infty} h_n = 0$.}
   \end{reponses}
\end{questionmult}
}
%%%%%%%%%%%%%%%%%%%%%%%%%%%%%%%%%%%%%%%%%%%%%%%%%%%%%%%%%%%%%%%%%%%%%%%%%%%%
\element{quNonClassee}{
\begin{question}{} Pour les signaux aléatoires:
  \begin{reponses}
    \bonne{ la stationnarité au sens strict entraîne la stationnarité
    au sens large.}
    \mauvaise{ la stationnarité au sens strict entraîne l'ergodicité.}
    \mauvaise{ la stationnarité au sens large entraîne la stationnarité
    au sens strict.}
    \mauvaise{ la stationnarité au sens large et au sens strict
    entraînent l'ergodicité.}
  \end{reponses}
\end{question}
}
%%%%%%%%%%%%%%%%%%%%%%%%%%%%%%%%%%%%%%%%%%%%%%%%%%%%%%%%%%%%%%%%%%%%%%%%%%%%
\element{quNonClassee}{
\begin{question}{} Soit un filtre temps continu de réponse impulsionnelle
$h(t)$ et de réponse en fréquence $H(f)$. Le filtre est excité en
entrée par un signal \emph{aléatoire} $x(t)$ stationnaire au sens
large et sa sortie est notée $y(t)$.
\begin{reponses}
  \mauvaise{ Les transformées de Fourier $X(f)$ et $Y(f)$ des signaux
  aléatoires $x(t)$ et $y(t)$ sont liées par $Y(f)=H(f)X(f)$.}
  \mauvaise{ Les transformées de Fourier $X(f)$ et $Y(f)$ des signaux
  aléatoires $x(t)$ et $y(t)$ sont liées par $Y(f)=H(f)\star X(f)$ où
  $\star$ représente la convolution.}
  \mauvaise{ Les transformées de Fourier $X(f)$ et $Y(f)$ des signaux
  aléatoires $x(t)$ et $y(t)$ sont égales aux densités spectrales de
  puissance respectives et on a $Y(f)=|H(f)|^2X(f)$.}
  \bonne{ On ne peut pas définir de transformée de Fourier des signaux
  aléatoires $x(t)$ et $y(t)$ dans le sens usuel (càd tel que
  rencontré en cours de mathématiques de début
  d'année).} % En revanche, les
  % densités spectrales de puissance notées $\Gamma_x(f)$ et
  % $\Gamma_y(f)$ respectivement en entrée et en sortie sont liées par:
  % $\Gamma_y(f)=|H(f)|^2\Gamma_x(f)$.
\end{reponses}
\end{question}
}
%%%%%%%%%%%%%%%%%%%%%%%%%%%%%%%%%%%%%%%%%%%%%%%%%%%%%%%%%%%%%%%%%%%%%%%%%%%%
\element{quNonClassee}{
\begin{question}{} Deux séances de travaux pratiques (TP) ont eu lieu.
\begin{reponses}
  \bonne{Les deux séances étaient sous \matlab. L'une concernait
  l'analyse spectrale numérique et l'autre le filtrage numérique.}
  \mauvaise{Les deux séances étaient sous \matlab. L'une concernait
  l'analyse spectrale numérique et l'autre l'analyse spectrale
  analogique.}
  \mauvaise{Une séance était sous \matlab et traitait d'analyse spectrale
  numérique. L'autre séance utilisait un analyseur de spectre
  analogique.}
  \mauvaise{Une séance était sous \matlab et traitait d'analyse spectrale
  analogique. L'autre séance utilisait un analyseur de spectre
  analogique et traitait de filtrage.}
\end{reponses}
\end{question}
}
%%%%%%%%%%%%%%%%%%%%%%%%%%%%%%%%%%%%%%%%%%%%%%%%%%%%%%%%%%%%%%%%%%%%%%%%%%%%
\element{quNonClassee}{
\begin{question}{} \label{qu:Matlab1} La figure \ref{fig:MatlabSpec1} sur la
   page \pageref{fig:MatlabSpec1} a été obtenue à l'aide du logiciel
   \textsc{Matlab}. Indiquer la ou les séquences de commandes qui
   peuvent donner ce tracé.
   \begin{reponses}
   \bonne{ \texttt{t = [1:512]; x=cos(2*pi*0.15*t); plot(abs(fft(x)));}}
   \mauvaise{ \texttt{t = [1:512]; x=cos(2*pi*0.15*t); plot(fft(x));}}
   \mauvaise{ \texttt{t = [1:512]; x=cos(2*pi*0.15*t);
       plot((0:511)/512,abs(fft(x)));}}
   \mauvaise{ \texttt{t = [1:512]; x=cos(2*pi*0.15*t); plot((0:511)/512,fft(x));}}
   \end{reponses}
 \end{question}
}


